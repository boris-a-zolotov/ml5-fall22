\subsection{Лекция 9}

\subsubsection{Игры Эренфойхта}

Считаем, что сигнатура $\sigma$ не содержит функциональных символов и конечна. Первое требование не существенно, ведь можно рассматривать соответствующий предикат. $\ba$ и $\bb$ — $\sigma$-структуры, $n$ — произвольное натуральное число.

\begin{definition}
    В \emph{игре Эренфойхта с $n$ ходами} $G_n(\ba, \bb)$ каждый из двух игроков I и II  делает по $n$ ходов. Сначала I выбирает элемент $A\cup B$ ($A, B$ — множества, соответствующие структурам $\ba, \bb$). Затем II выбирает элемент в другой структуре. Получили <<обогащение>> каждой структуры константой: $(\ba, a), (\bb, b)$. Такая пара ходов повторяется n раз.

    II выигрывает в партии, если конечные подструктуры, порождённые $\overline{a}, \overline{bb}$ (в каждом наборе по n элементов) изоморфны относительно изоморфизма, который отправляет $a_i \mapsto b_i$ и для каждой константы $c^{\mathbb{A}}\to c^{\mathbb{B}}$
\end{definition}

\begin{remark}
    Игроков в такой игре иногда называют Новатор и Консерватор. В англоязычной литературе — spoiler и duplicator или $\forall$ (Adam) и $\exists$ (Eve).
\end{remark}

\begin{definition}
    В игре $G(\ba, \bb)$ в свой первых ход I выбирает натуральное число n, после чего игра идёт как $G_n(\ba, \bb)$.
\end{definition}

% QUESTION: частичная?
Обозначим через $S_1$ стратегию для первого игрока (функцию $(A\cdot B)^*\to A\cup B$), через $S_2$ стратегию для второго игрока (функции $(A\cdot B)^*\cdot A\to B$ и $(A\cdot B)^*\cdot B\to A$). Стратегия $S_i$ для игрока $i$ называется выигрышной, если он, следуя этой стратегии, выигрывает вне зависимости от стратегии другого игрока.

Обозначим $G_n^I(\ba, \bb)$ наличие выигрышной стратегии у игрока I в игре $G_n(\ba, \bb)$. Аналогично определяется $G_n^{II}$.

\begin{prop}
    Свойства игр и стратегий
    \begin{enumerate}
        \item \label{9_game_prop_1} $G_{n+1}^I(\ba, \bb) \Leftrightarrow \left( \exists a\in\ba~\forall b\in\bb~G_n^I((\ba, a), (\bb, b))\right)
                                    \vee\left( \exists b\in\bb~\forall a\in\ba~G_n^I((\ba, a), (\bb, b))\right) $
        \item \label{9_game_prop_2} $G_{n+1}^{II}(\ba, \bb) \Leftrightarrow \left( \forall a\in\ba~\exists b\in\bb~G_n^{II}((\ba, a), (\bb, b))\right)
                                                \wedge\left( \forall b\in\bb~\exists a\in\ba~G_n^{II}((\ba, a), (\bb, b))\right) $

        \item \label{9_game_prop_3} $G^{II}(\ba, \bb) \Leftrightarrow \forall n~G_n^{II}(\ba, \bb)$
        \item \label{9_game_prop_4} $G^{I}(\ba, \bb) \Leftrightarrow \exists n~G_n^{I}(\ba, \bb)$

        \item \label{9_game_prop_5} В $G_n(\ba, \bb)$ ровно один из игроков имеет выигрышную стратегию.
        \item \label{9_game_prop_6} В $G (\ba, \bb)$ ровно один из игроков имеет выигрышную стратегию.
    \end{enumerate}
\end{prop}
\begin{proof}
    В пункте \ref{9_game_prop_1} выигрышная стратегия у первого игрока есть либо когда есть выигрышный первый ход в первую структуру, либо выигрышный первый ход во вторую структуру. Формально утверждение доказывается по индукции. Пункт \ref{9_game_prop_2} доказывается аналогично.

    В \ref{9_game_prop_5} достаточно проверить, что оба игрока не могут иметь выигрышную стратегию. Действительно, если такие стратегии $S_1, S_2$ есть, то можно запустить партию с этими стратегиями. В итоге одна из них проиграет.
\end{proof}

\begin{definition}
    \emph{Квантовая глубина} формулы $\vfi$ — натуральное число $q(\vfi)$, определяемое рекурсией по $\vfi$:
    \begin{enumerate}
        \item Если $\vfi$ — простейшая, то $q(\vfi)=0$.
        \item Если $\vfi=\neg\vfi_1$, то $q(\vfi)=q(\vfi_1)$.
        \item Если $\vfi=\vfi_1\wedge\vfi_2$, то $q(\vfi)=\max(q(\vfi_1), q(\vfi_2))$.
        \item Если $\vfi=\exists x~\vfi_1$, то $q(\vfi) = q(\vfi_1)+1$.
    \end{enumerate}
\end{definition}

\begin{definition}
    Обозначим через $C_n^{\overline{x}}$ множество всех $\sigma$-формул $\vfi(\overline{x})$ глубины не более $n$.
\end{definition}

\begin{prop}
    Множество $C_n^{\overline x}/{\equiv}$ (отношение $\equiv$ обозначает равносильность формул) конечно.
\end{prop}

\begin{proof}
    % Я не знаю формальное определение булевых комбинаций. Но и так понятно.
    Доказываем индукцией по $n$. База индукции $n=0$. $C_0^{\overline{x}}$ — бескванторные формулы. Каждая формула оттуда эквивалентна булевой комбинации формул вида $x_i=x_j, x_i=c, c=x_i, c=d$ ($c, d$ — константы)и $P(t_1, \dots, t_m)$ ($t_i$ — переменные или константы). Таких формул конечное число, поэтому и их булевых комбинаций конечное число.

    Переход: рассматриваем $n>0$. В этом случае $C_n^{\overline{x}} = C_{n-1}^{\overline{x}}\cup D_n^{\overline{x}}$, где $D_n^{\overline{x}}$ — формулы с факторной глубиной ровно $n$. Фактор первого множетсва конечен по предположению индукции, а для второго выполнено
    $$D_n^{\overline{x}}/{\equiv} = BC(\{\exists y~\psi(\overline{x}, y)\mid \psi\in D_{n-1}^{\overline{x}}\}),$$
    что тоже конечно. % поформальнее? 
\end{proof}

\begin{theorem}
    $G_n^{II} \iff \forall \vfi\in C_n (\vfi^{\mathbb A} = \vfi^{\mathbb B})$
\end{theorem}
\begin{proof}
    Будем доказывать обобщение этой теоремы: пусть $\overline{a}, \overline{bb}$ — наборы элементов одинаковой длины, равные по длине набору $\overline{x}$; докажем
    $$G_n^{II}(\ba, \bb) \iff \forall \vfi\in C_n (\vfi^{\mathbb A}=\vfi^{\mathbb B}).$$

    TODO: доказательство
\end{proof}
