\subsection*{Лекция 15}

\begin{proof}
    % TODO: пару строчек написать всё-таки можно
    Доказательство в одну сторону рассматривалось на практике.

    В другую сторону: доказываем, что любая R-вычислимая функция является рекурсивной. Рассматриваем тотальную функцию $\vfi_P(\overline{x})$. Пусть $a = \cor{P}$. Введём вспомогательную функцию
    \[
        \begin{aligned}
            g(\overline{x}) &= \mu t~(\text{$s_t$ заключительное}) \\
                            &= \mu t~(\beta(\text{сос}(a, \overline{x}, t), \text{пам}(a)+2)\ge \text{дл}(a)).
        \end{aligned}
    \]
    Она рекурсивна и вычисляет номер первого момента, в который программа прекратит работу. Отсюда получаем рекурсивность нужной функции:
    $$
    \phi_P(\overline{x}) = \beta(\text{сос}(a, \overline{x}, g(\overline{x})), 1)
    $$
\end{proof}

То же самое доказательство работает для другого варианта теоремы:
\begin{theorem}
    Класс рекурсивных частичных функций совпадает с классом R-вычислимых частичных функций.
\end{theorem}

%%%%%%%%%%%%%%
\begin{definition}
    Для произвольной (возможно не вычислимой) функции $h:\bn\rightarrow\bn$
    \begin{itemize}
        \item \emph{Частичные функции, вычислимые относительно $h$} определяются так же, как и обычные рекурсивные частичные функции, но в списке начальных функций присутствует $h$.
        \item \emph{R-вычислимые относительно $h$ частичные функции} (или \emph{с оракулом $h$}) — функции, вычислимые R-программами с оракулом $h$.
        \item \emph{R-программа с оракулом $o$} — R-программа с дополнительным оператором $r_i = o(r_i)$.
    \end{itemize}
\end{definition}

Приведём ещё один вариант теоремы:
\begin{theorem}
    Для любой функции $h:\bn\rightarrow\bn$ класс частичных функций, рекурсивных относительно $h$ совпадает с классом функций, R-вычислимых относительно $h$.
\end{theorem}


\subsubsection{Главная вычислимая нумерация рекурсивных частичных функций}
\begin{definition}
    $\Phi$ — класс одноместных рекурсивных частичных функций. Нумерация $\nu:\bn\rightarrow\Phi$ называется вычислимой, если двуместная функция $\tilde{\nu}(n, x) = \nu_n(x)$ вычислима.
\end{definition}

\begin{definition}
    Вычислимая нумерация $\nu$ называется главной, если любая вычислимая нумерация $\mu:\bn\rightarrow\Phi$ сводится к $\nu$, то есть $\mu = \nu\circ f$ для некоторой рекурсивной функции $f$.
\end{definition}

\begin{exmpl}
    Основным рассматриваемым примером вычислимой нумерации является $\vfi$, где
    \[
        \vfi_n = 
        \begin{cases}
            \vfi_P^{(1)}, & \text{если $n = \cor{P}$} \\
            \emptyset, & \text{иначе}
        \end{cases}
    \]
\end{exmpl}

% TODO: (стиль) названия теорем должны быть с маленькой буквы?
\begin{theorem}[o главной вычислимой нумерации]
    $\vfi$ — главная вычислимая нумерация одноместных рекурсивных частичных функций.
\end{theorem}
\begin{proof}
    Сначала проверим вычислимость $\tilde{\vfi}$. Модифицируем $g$ из предыдущего раздела, добавив ко входу код программы:
    $$
    g(n, x) = \mu t~(\text{Прог}(n)\wedge\beta(\text{сос}(n, x, t), \text{пам}+2)\ge\text{дл}(n)).
    $$
    Тогда $\tilde{\vfi}(n, x) = \beta(\text{сос}(n, x, g(n, x)), 1)$ рекурсивна.

    Теперь проверим, что $\vfi$ — главная. Зафиксируем вычислимую нумерацию $\mu$. Обозначим за M R-программу, которая вычисляет $\tilde{\mu}(n, x)$. Для всех $n$ построим программы $P_n$, для которых $\mu_n = \vfi_{P_n}^{(1)}$.

    Идея программы — получив на входе $x$, дописать $n$, после чего запустить программу M. Код выглядит так:

    % TODO: (стиль) оформить код покрасивее
    \begin{algorithm}
        $r_1 \defeq r_0$ \\
        $r_0 \defeq 0 $\\
        $r_o \defeq r_0+1$ \\
        \dots \\
        $r_o \defeq r_0+1$ \\ 
        M(n+2) 
    \end{algorithm}
    (Последняя строчка обозначает код программы M, где все индексы увеличены на $n+2$). Нетрудно проверить, что функция $s(n) \defeq \cor{P_n}$ является рекурсивной. Значит $\mu = \vfi\circ s$.
\end{proof}

\begin{theorem}[о неподвижной точке]
    Для любой одноместной рекурсивной функции $f(x)$ найдётся $e$, для которого $\vfi_e = \vfi_{f(e)}$
\end{theorem}
\begin{proof}
    Рассмотрим вычислимую нумерацию $\mu_n = \vfi_{\vfi_n(n)}$. По предыдущей теореме, существует рекурсивная функция $s$ такая, что $\mu_n = \vfi_{s(n)}$. $f\circ s$ рекурсивная, поэтому $\exists v: f\circ s = \vfi_v$. Значит $e = s(v)$ подходит:
    $$ \vfi_{s(v)} = \mu_v = \vfi_{\vfi_v(v)} = \vfi_{f(s(v))}$$
\end{proof}

% TODO: можно написать, что используется для построения квайнов

\begin{theorem}[Райса о неразрешимости свойств рекурсивных частичных функций]
    Пусть $\emptyset\subsetneq C\subsetneq \Phi$ (собственного подкласса $\Phi$) множество $\vfi^{-1}(C) = \{n\mid \vfi_n\in C\}$ нерекурсивно.
\end{theorem}
\begin{proof}
    Предположим, что это множество рекурсивно. Возьмём $a\in \vfi^{-1}(C)$ и $b\in \bn \textbackslash \vfi^{-1}(C)$. Рассмотрим рекурсивную функцию 
    $$
    f(x) = \begin{cases}
        a, & \text{если $x\notin \vfi^{-1}(C)$}\\
        b, & \text{если $x\in \vfi^{-1}(C)$}
    \end{cases}
    $$
    По теореме о неподвижной точке, для некоторого $c$ выполнено $\vfi_c = \vfi_{f(c)}$. Но одна функцию лежит в классе $C$, а другая — нет. Противоречие.
\end{proof}


%%%%%%%%%%%%
%%%%%%%%%%%%
\subsubsection{Рекурсивно перечислимые множества. Сводимости. Тьюрингов скачок.}

Напоминание: $A\subset \bn$ рекурсивно перечислимо, если $A=\emptyset\vee A = rng(f)$ для некоторой рекурсивной функции $f$.

% TODO: проверить, где вычислимая, а где перечислимая
\begin{definition}
    Нумерация $\nu: \bn\rightarrow \mathcal{E}$ всех рекурсивно перечислимых множеств называется \emph{вычислимой}, если $\{(n, x)\mid x\in\nu_n \}$ вычислимо (TODO: перечислимо?). Вычислимая нумерация называется главной, если к ней сводится любая другая вычислимая нумерация.
\end{definition}
\begin{prop}\
    \begin{enumerate}
        \item A рекурсивно $\Leftrightarrow$ A и $\overline{A}$ рекурсивно перечислимы (теорема Поста)
        \item А рекурсивно перечислимо $\Leftrightarrow$ $A=rng(\vfi_n)$ для некоторого $n\in \bn$ $\Leftrightarrow$ $A = dom(\vfi_n)$ для некоторого $n\in\bn$
        \item $W_n = dom(\vfi_n)$ — главная вычислимая нумерация класса $\mathcal{E}$ всех рекурсивно перечислимых множеств.
        \item $C = \{n \mid b\in W_n\}$ рекурсивно перечислимо, но не рекурсивно.
        \item Любое рекурсивно перечислимое множество m-сводится к C.
    \end{enumerate}
\end{prop}

\begin{proof}
    % Первый пункт был на практике

    % TODO: как делать ref к пункту enumeration?
    Доказательство 4 пункта: предположим, что С рекурсивно. Тогда $\overline{C}$ тоже рекурсивно, а значит и рекурсивно перечислимо. Значит $\overline{X} = W_e$ для некоторого $e$. Тогда $e\in W_e \Leftrightarrow  e\notin W_e$. Противоречие.

    % TODO: тут был какой-то пример про [MA] и [ПА]
\end{proof}

\begin{definition}
    Мы говорим, что \emph{$f$ сводится по Тьюрингу к $g$}, если $f$ рекурсивна относительно $g$. Обозначается $f\le_T g$. Для множеств $A\le_T B \Leftrightarrow \chi_A \le_T \chi_B$ 
\end{definition}

\begin{definition}
    \emph{Тьюринговым скачком} множества $A\subset \bn$ называют $A'=\{n\mid n\in W_n^A\}$, где $W_n^A = dom(\vfi_n^A)$ (A — оракул).
\end{definition}

% TODO: свойства тьюрингова скачка.
