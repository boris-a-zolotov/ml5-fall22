\documentclass[a4paper,11pt]{article}
\usepackage{../ml5}

\begin{document}

\begin{center}
	{\Large Домашнее задание 6. Элиминация кванторов.}

	{\it (12 октября\ \(\to\) 19 октября)}
\end{center}

\begin{enumerate}

\item \begin{enumerate}
	\item[(а)] Докажите, что структура $(\mathbb{Z};=,0,S)$, где  $S(x)=x+1$, допускает элими-\linebreak нацию кванторов.
	\item[(б)] Опишите определимые отношения в структуре $(\mathbb{Z};=,0,S)$.
	\item[(в)] Определимо ли отношение $<$ в этой структуре?
\end{enumerate}

\item \begin{enumerate}
	\item[(а)] Докажите, что теория плотного линейного порядка без наименьшего и наибольшего элемента допускает элиминацию кванторов.
	\item[(б)] Опишите определимые отношения в структуре $(\mathbb{Q};=,<)$.
	\item[(в)] Какие элементы определимы в структуре $(\mathbb{Q};=,<)$?
\end{enumerate} \medskip

\item \begin{enumerate}
	\item[(а)] Докажите, что структура $(\mathbb{Z};=,S,<)$ допускает элиминацию кванторов.
	\item[(б)] Опишите определимые отношения в структуре $(\mathbb{Z};=,S,<)$.
	\item[(в)] Определима ли функция $S$  в структуре $(\mathbb{Z};=,<)$ бескванторной формулой? 
\end{enumerate} \medskip

\item \begin{enumerate}
	\item[(а)] Докажите, что теория плотного линейного порядка с наименьшим, но без наибольшего элемента $\{=,a,<\}$ допускает элиминацию кванторов.
	\item[(б)] Опишите определимые отношения в структуре $\left(\left[0,1\right);\ =,0,<\right)$.
\end{enumerate} \medskip

\item \begin{enumerate}
	\item[(а)] Докажите, что структура $(\mathbb{N};=,0,S)$, где  $S(x)=x+1$, допускает элиминацию кванторов.
	\item[(б)] Опишите определимые отношения в структуре $(\mathbb{N};=,0,S)$. 
\end{enumerate}

\end{enumerate}

\end{document}
