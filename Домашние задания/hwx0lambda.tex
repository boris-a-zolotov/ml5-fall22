\documentclass[a4paper,11pt]{article}
\usepackage{../ml5,epigraph}

	\DeclareMathOperator{\Pow}{Pow}
	\DeclareMathOperator{\Pred}{Pred}

\begin{document}

   \newcommand{\enumsep}{\vspace{-2.8mm}
   		\begin{enumerate}[itemsep=0.4mm,leftmargin=2.5mm]}

   \renewcommand{\l}{\ensuremath{\lambda}}
   
   \newcommand{\ltrue}{\text{\bf True}}
   \newcommand{\lfalse}{\text{\bf False}}

\begin{center}
	{\Large Домашнее задание 10\l .\quad\l -исчисление.}

	{\it (Группы Б02 и Б03, 9 ноября\ \(\to\)\ 16 ноября)}
\end{center}

\begin{flushright}
  \begin{minipage}{7cm}
    \begin{verbatim}
  \item \begin{enumerate}
  \item[(а)]
    \end{verbatim} \vspace{-5.5mm} \hrule
  \end{minipage}

{\it Домашнее задание по логике}
\end{flushright}


\begin{enumerate}
	\item Напомним, \[\ltrue \coloneqq \l x.\ \l y.\ x,\qquad
	                 \lfalse \coloneqq \l x.\ \l y.\ y.\]

	Выразите логические операции \(\land\), \(\lor\), \(\oplus\), \(\rightarrow\), \(\downarrow\) (отрицание дизъюнкции) как \l -термы, которые можно применять к \(\ltrue\) и \(\lfalse\). Покажите эквивалентность термов, соответствующих правым и левым частям каких-нибудь пяти основных равносильностей.

	\item \begin{enumerate}
	   \item[(а)] Выпишите \l -терм \(\Pow\), соответствующий возведению в степень чёрчевских нумералов. Проверьте себя, получив \(\mathbf{8}\) в результате \(\beta\)-редукции терма\linebreak \(\Pow\ \mathbf{2}\ \mathbf{3}\). А что будет, если редуцировать \(\mathbf{2}\ \mathbf{2}\ \ldots\ \mathbf{2}\)?
	   \item[(й)] Выпишите \l -терм IsZero, который при применении к чёрчевскому нумералу \(n\) редуцируется в \(\ltrue\), если \(n = \mathbf{0}\), иначе в \(\lfalse\).
	\end{enumerate}
	   
\item Пусть \(\l x.\ \l y.\ \l f.\ \lr*{f\ x\ y}\) создаёт упорядоченную пару из элементов \(x\) и \(y\). Например, \(\l f.\ f\ 3\ 5\)~— такая вот упорядоченная пара. \begin{enumerate}
	\item[(а)] Выпишите \l -терм First, который при применении к упорядоченной паре вернёт её первый элемент.
	\item[(б)] Выпишите \l -терм, преобразующий пару \(\lr*{a,b}\) в пару \(\lr*{b,b+1}\). Отсюда получите \l -терм \(\Pred\), соответствующий взятию предыдущего чёрчевского нумерала (\(\Pred \mathbf{0} = \mathbf{0}\)). Проверьте себя, применив его к \(\mathbf{3}\) и редуцировав результат.
	\end{enumerate}

	\item Придумайте терм без нормальной формы: который можно бесконечное число раз \(\beta\)-редуцировать.

	\item Пусть \[\mathbf{S} \coloneqq \l x.\ \l y.\ \l z.\ \lr*{\lr*{x\ z}\ \lr*{y\ z}},\qquad
	              \mathbf{K} \coloneqq \ltrue.\]

Выразите, применяя \(\mathbf{S}\) и \(\mathbf{K}\) друг к другу, \(\lfalse\); \(\oplus\); {\it вашу желаемую оценку за курс логики.}
	
\end{enumerate}

\end{document}


