\documentclass[a4paper,11pt]{article}
\usepackage{../ml5}

\begin{document}

   \newcommand{\enumsep}{\vspace{-2.8mm}
   		\begin{enumerate}[itemsep=0.4mm,leftmargin=2.5mm]}

\begin{center}
	{\Large Домашнее задание 13. Проблемы разрешимости. R-Вычислимость.}

	{\it (30 ноября\ \(\to\)\ 7 декабря)}
\end{center}

Говорят, что $A\subseteq\mathbb{N}$ $m$-сводится к $B\subseteq\mathbb{N}$ (символически, $A\leq_mB$), если  $A=f^{-1}(B)$ для некоторой рекурсивной функции $f$.

\begin{enumerate}

\item Докажите, что отношение $m$-сводимости рефлексивно и транзитивно, а фактор-множество по индуцированному им отношению эквивалентности $\equiv_m$ континуально.

Докажите, что если $A\leq_mB$ и $B$ рекурсивно, то $A$ также рекурсивно.

Докажите, что  множество всех натуральных чисел не определимо в поле вещественных чисел, а также в поле комплексных чисел. 

\item Выясните, какие соотношения по $m$-сводимости существуют между следующими множествами предложений (точнее, между соответствующими множествами кодов): $Th(\mathbb{N})$, $Th(\mathbb{Z})$, $Th(\mathbb{Q})$,  $Th(\mathbb{R})$  $Th(\mathbb{C})$, арифметика Пеано. Все указанные теории расссматриваются в сигнатуре $\sigma=\{=,+,\cdot,0,1\}$.  

\item Докажите, что следующие функции R-вычислимы: 

постоянные функции, $I^n_k$, $+$, $\chi_\leq$, $max\{x,y\}$, $\cdot$, $x^y$, $x!$. 

\item Докажите, что суперпозиция R-вычислимых  функций R-вычислима.

\item Докажите, что минимизация R-вычислимой функции R-вычислима.

\end{enumerate}




\end{document}


