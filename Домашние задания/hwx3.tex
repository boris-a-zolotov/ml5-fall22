\documentclass[a4paper,11pt]{article}
\usepackage{../ml5}

\begin{document}

   \newcommand{\enumsep}{\vspace{-2.8mm}
   		\begin{enumerate}[itemsep=0.4mm,leftmargin=2.5mm]}

\begin{center}
	{\Large Домашнее задание 13. Проблемы разрешимости. R-Вычислимость.}

	{\it (30 ноября\ \(\to\)\ 7 декабря)}
\end{center}

Говорят, что $A\subseteq\mathbb{N}$ $m$-сводится к $B\subseteq\mathbb{N}$ (символически, $A\leq_m B$), если  $A=f^{-1}(B)$ для некоторой рекурсивной функции $f$.

{\it Язык программирования R:} программа представляет из себя список команд, пронумерованных от 0 до \(N\). Также есть счётный набор переменных \(r_0, r_1, r_2, \ldots\)

Команды выполняются последовательно. Каждая команда~— оператор присвоения или условный оператор. \begin{enumerate}
	\item[(\(*\))] Оператор присвоения имеет один из трёх видов: \[r_n \coloneqq 0,\quad r_n \coloneqq r_n+1,\quad r_n\coloneqq r_m.\]
	\item[(\(*\))] Условный оператор имеет вид \(r_n = r_m\ \Rightarrow\ k\), где \(k\)~— номер команды, к которому переходит программа, если условие выполнилось.
\end{enumerate}

Программа останавливается, если была выполнена последняя команда или условный оператор отправил нас в команду, номер которой больше \(N\). Программа вычисляет функцию \(\varphi \lr*{x_0,\ldots,x_n}\) от \(n+1\) переменной, если перед началом её работы
	\[r_i = x_i;\quad r_{n+1}, r_{n+2}, \ldots = 0,\]
а после окончания \(r_0\) (или, если вам угодно, \(r_{n+1}\)) равно \(\varphi \lr*{x_0,\ldots,x_n}\).

\begin{enumerate}

\item[1 а)] \stepcounter{enumi} \enumsep
	\item[ ] Докажите, что отношение $m$-сводимости рефлексивно и транзитивно, а фактор-множество по индуцированному им отношению эквивалентности $\equiv_m$ континуально.
	\item[(б)] Докажите, что если $A\leq_mB$ и $B$ рекурсивно, то $A$ также рекурсивно.
	\item[(в)] Докажите, что  множество всех натуральных чисел не определимо в поле вещественных чисел, а также в поле комплексных чисел. \end{enumerate}

\item Выясните, какие соотношения по $m$-сводимости существуют между следующими множествами предложений (точнее, между соответствующими множествами кодов): $Th(\mathbb{N})$, $Th(\mathbb{Z})$, $Th(\mathbb{Q})$,  $Th(\mathbb{R})$, $Th(\mathbb{C})$, арифметика Пеано. Все указанные теории рассматриваются в сигнатуре $\sigma=\{=,+,\cdot,0,1\}$.  

\item Докажите, что существуют программы на R, вычисляющие следующие функции: \vspace{-9.5mm}

\begin{center}
	постоянные функции,\quad
	$I^n_k$,\quad $+$,\quad 
	$\chi_\leq$,\quad $max\{x,y\}$,\quad 
	$\cdot$,\quad $x^y$,\quad $x!$. 
\end{center}

\item Докажите, что минимизация; суперпозиция функций, для которых есть программы на R, имеют программы на R.

\item Докажите, что любая программа на R может быть представлена \(\l\)-термом.

\end{enumerate}

\end{document}
