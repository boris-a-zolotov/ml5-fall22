\documentclass[a4paper,11pt]{article}
\usepackage{../ml5}

\begin{document}

\begin{center}
	{\Large Домашнее задание 8. Выводимость.}

	{\it (26 октября\ \(\to\)\ 2 ноября)}
\end{center}

\begin{enumerate}
	\item Докажите, что “основные равносильности логики предикатов” (т.~е. равносильности, используемые при приведении формул к ДНФ и к предваренному виду) выводимы в двух версиях гильбертовского ИП и в генценовском исчислении секвенций.

	\item Докажите, что “основные тавтологии” (т.~е. аксиомы гильбертовского ИП, не содержащие кванторов в явном виде), выводимы в генценовском исчислении секвенций. 
\end{enumerate}

Докажите, что выводимы — то есть, сначала объясните “на пальцах”, а затем попытайтесь формализовать в одной из систем (Генценовской или Гильбертовской):

\begin{enumerate} \setcounter{enumi}{2}
	\item Из аксиом упорядоченных абелевых групп~— формулы, выражающие следующие утверждения: \begin{enumerate}
	\item[(а)] если элемент $x$ положителен, то элементы $x+x$ и $x+x+x$ также положительны;
	\item[(б)] если элемент $x$ отрицателен, то элементы $x+x$ и $x+x+x$ также отрицательны;
	\item[(в)] квадрат любого ненулевого элемента положителен;
	\item[(г)] порядок является плотным. 
	\end{enumerate}

	\item Из аксиом арифметики Пеано~— формулы, выражающие следующие утверждения: \begin{enumerate}
	\item[(а)] сложение ассоциативно и коммутативно;
	\item[(б)] умножение ассоциативно и коммутативно;
	\item[(в)] умножение дистрибутивно относительно сложения;
	\item[(г)] между $x$ и $x+1$ нет других элементов.
	\end{enumerate}

	\item Из аксиом ZFC~— формулы, выражающие следующие утверждения: \begin{enumerate}
	\item[(а)] существует единственное пустое множество;
	\item[(б)] для любых двух множеств существуют и единственны их объединение, пересечение, и разность;
	\item[(в)] две упорядоченные пары (в смысле \(\set*{\set*{a}, \set*{a,b}}\)) равны в точности тогда, когда равны их первые и вторые компоненты;
	\item[(г)] если  $x$~— ординал, то  $x+1=x\cup\{x\}$ тоже ординал, и между ними нет дру-\linebreak гих ординалов;
	\item[(д)] существует единственное множество натуральных чисел (определяемых как ординалы специального вида).
	\end{enumerate}
\end{enumerate}






\end{document}


