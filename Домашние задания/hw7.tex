\documentclass[a4paper,11pt]{article}
\usepackage{../ml5}

\begin{document}

\begin{center}
	{\Large Домашнее задание 7. Игры Эренфойхта и\\[1.5mm] элементарная эквивалентность.}

	{\it 19 октября\ \(\to\) 26 октября} \\
	{\it Последнее в разделе “Логика предикатов”}
\end{center}

\begin{enumerate}
	\item Пусть \(\bn\), \(\bz\), \(\bq\), \(\br\)~— структуры сигнатуры $\set*{ =,\ < }$ с обычным отношением порядка. С помощью игр Эренфойхта обоснуйте, какие из этих сруктур элементарно эквивалентны, а какие нет. В последнем случае найдите наименьшее $n$, для которого первый игрок имеет выигрышную стратегию в соответствующей $n$-игре Эренфойхта, и опишите эту стратегию.

	\item Решите задачу 1 для структур \(\bn\), \(\bn+\bn\), \(\bn+\bq\), \(\bn+\bz\), где сумма двух порядков определяется как их дизъюнктное объединение, в котором каждый элемент первого порядка меньше каждого элемента второго. 

	\item Решите задачу 1 для структур \(\bn\), \(\bn \times \bn\), \(\bn \times \bz\), \(\bn \times \lr*{\bn + \bn}\), где $\times$ есть декар-\linebreak тово произведение структур, а порядок определён лексикографически:
	\[ (x,y) < (p,q)\ \ \Longleftrightarrow\ \ 
	   \lr*{x<p}\ \lor\ \lr*{x=p\ \land\ y<q}.\]

	\item Докажите, что для любого $n$ найдется $m$ такое, что любые два линейных порядка с более чем $m$ элементами $n$--элементарно эквивалентны.

	Выведите из этого, что не существует $\set*{ =,\ < }$--предложения, которое истинно на всех конечных линейных порядках с четным числом элементов и ложно на всех конечных линейных порядках с нечетным числом элементов.

	\item Структуры сигнатуры $\set*{ =,\ <,\ P }$ ($P$~— одноместный предикатный символ) на множестве $\bn$ естественным образом отождествляются с бесконечными двоичными словами (т.~е.~с последовательностями битов). Докажите, что не существует $\set*{ =,\ <,\ P }$--предложения, истинного в точности на бесконечных двоичных словах с четным числом единиц. 

\end{enumerate}

\end{document}
