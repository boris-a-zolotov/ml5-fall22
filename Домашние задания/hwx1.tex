\documentclass[a4paper,11pt]{article}
\usepackage{../ml5}

\begin{document} \newgeometry{top=16mm,bottom=16mm}

   \newcommand{\enumsep}{\vspace{-2.8mm}
   		\begin{enumerate}[itemsep=0.4mm,leftmargin=2.5mm]}

\begin{center}
	{\Large Домашнее задание 9. Свойства выводимости.}

	{\it (2 ноября\ \(\to\)\ 9 ноября)}
\end{center}


\begin{center}
Задание 10. Рекурсивные функции. 
\end{center}

Если предикат $P(x)$ истинен при
некотором значении $x \in \mathbb{N}$, то $\mu x P(x)$ --- наименьшее
число из $\mathbb{N}$, для которого предикат $P(x)$ истинен. Например,
$\mu x(4 < x^2) = 3$.

 При $n > 0$ и $1 \leq k \leq n$ определим
$n$-местную функцию $I_n^k$ следующим образом: $I_n^k(x_1,\ldots, x_n)
= x_k$, т. е. $I_k^n$ из $n$ аргументов выбирает тот, который
стоит на месте $k$. Введем еще двухместную функцию $l(x, y)$:
$l(x, y) = 0$ при $x < y$ и $l(x, y) = 1$ при $x\geq y$.

 Семейство рекурсивных функций определим индуктивно следующим образом:
Функции $+$, $\cdot$, $l$ и $I_n^k$ рекурсивны; если рекурсивны
функции $g(y_1,\ldots, y_k)$, $h_1(\bar{x}),\ldots, h_k(\bar{x})$,
то функция $g(h_1(\bar{x}),\ldots, h_k(\bar{x}))$ рекурсивна; если
функция $g(\bar{x}, y)$ рекурсивна и $\forall\bar{x} \exists y
(g(\bar{x}, y) = 0)$, то функция $f(\bar{x}) = \mu y (g(\bar{x},
y) = 0)$ рекурсивна; других рекурсивных функций нет.

Характеристической функцией предиката $P(\bar{x})$ называют
функцию $\chi_{P}(\bar{x})$, задаваемую условиями:
$\chi_{P}(\bar{x}) = 0$, если $P(\bar{x}) =\text{И}$;
$\chi_{P}(\bar{x}) = 1$, если $P(\bar{x}) =\text{Л}$.
Предикат рекурсивен, если его характеристическая функция
рекурсивна.

1. Покажите, что предикат $<$ рекурсивен, и что всякая рекурсивная функция вычислима (т.е. существует вычисляющий ее алгоритм).
\medskip

2. Докажите, что: 

Если предикат $P(y_1,\ldots, y_k)$ и функции
$h_1(\bar{x}),\ldots, h_k(\bar{x})$ рекурсивны, то рекурсивен и
предикат $P(h_1(\bar{x}),\ldots, h_k(\bar{x}))$. 

Если
предикат $P(\bar{x}, y)$ рекурсивен и $\forall\bar{x} \exists y
P(\bar{x}, y)$, то функция $f(\bar{x}) = \mu y P(\bar{x}, y)$
рекурсивна. 
\medskip

3. Докажите, что: 

Если предикаты $P(\bar{x})$, $Q(\bar{x})$ и
$R(\bar{x}, y)$ рекурсивны, то предикаты $P(\bar{x}) \wedge
Q(\bar{x})$, $P(\bar{x}) \vee Q(\bar{x})$, $P(\bar{x}) \rightarrow
Q(\bar{x})$, $\neg p(\bar{x})$, $\forall y < z R(\bar{x}, y)$
и $\exists y < z R(\bar{x}, y)$ рекурсивны. 

Пусть
$P_1(\bar{x}),\ldots, P_k(\bar{x})$ --- рекурсивные предикаты
такие, что для любого истинен ровно один из этих предикатов, а
$g_1(\bar{x}),\ldots, g_k(\bar{x})$ --- рекурсивные функции. Тогда
рекурсивна и функция 

$f(\bar{x})~=~
\begin{cases}
g_1(\bar{x}),&\text{если} P_1(\bar{x})=\text{И}\\
\ldots\ldots\ldots\\
g_k(\bar{x}),&\text{если} P_k(\bar{x})=\text{И}
\end{cases}$ 
\medskip

4. Докажите рекурсивность следующих функций: $max\{x,y\}$; $[\sqrt{x}]$; $[x\sqrt{2}]$; $|x-y|$; $max\{0,x-y\}$; $f(x)=n$, если $x=2n$, и $f(x)=0$ в противном случае; $div(x,y)$ --- частное от деления $x$ на $y$ ($div(x,0) = 0$). 
\medskip

5. Докажите рекурсивность следующих функций: 
$rest(x,y)$ --- остаток от деления $x$ на $y$ ($rest(x,0) = x$);
НОД$(x,y)$ --- наибольший общий делитель $x$ и $y$ (НОД$(0,0) = 0$);
НОК$(x,y)$ --- наименьшее общее кратное $x$ и $y$ (НОК$(x,0) =$НОК$(0,y) = 0$).





\pagebreak


\begin{center}
Задание 11. Рекурсивность и вычислимость. 
\end{center}

1. Докажите, что множество натуральных чисел разрешимо тогда и только тогда, когда оно само и его дополнение перечислимы. 
\medskip

2. Докажите, что следующие теории перечислимы: 

множество всех логических следствий перечислимой теории конечной сигнатуры; 

множество всех логических следствий аксиом бесконечных полугрупп, групп, колец, и полей;

множество всех логических следствий аксиом арифметики Пеано;

множество всех логических следствий аксиом ZFC. 
\medskip

3. Докажите, что следующие теории разрешимы: 

множество всех логических следствий аксиом плотного линейного порядка без наибольшего и наименьшего элементов; 

множество всех логических следствий аксиом нетривиальных делимых абелевых групп без кручения;

множество всех логических следствий аксиом алгебраически замкнутых полей фиксированной характеристики;

множество всех логических следствий аксиом вещественно замкнутых упорядоченных полей;

множество всех логических следствий аксиом любой полной перечислимой теории.
\medskip

4. Докажите, что следующие теории разрешимы: 

множество всех логических следствий аксиом плотного линейного порядка; 

множество всех логических следствий аксиом алгебраически замкнутых полей;

множество всех логических следствий аксиом линейных порядков, булевых алгебр, абелевых групп (очень трудная задача, задача 4 считается решенной даже без этого пункта).
\medskip

5. Докажите аналоги утверждений 1 --- 4 с заменой разрешимости на рекурсивность и перечислимости на рекурсивную перечислимость (множеств кодов предложений).

\end{document}


