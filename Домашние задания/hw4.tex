\documentclass[a4paper,11pt]{article}
\usepackage{../ml5}

\begin{document}

\begin{center}
	{\Large Домашнее задание 4. Аксиоматизируемые классы}

	{\it (28 сентября\ \(\to\) 5 октября)}
\end{center}

\begin{enumerate}
	\item Пусть $\bn$ --- стандартная структура натуральных чисел в сигнатуре $\{=,+,\cdot\}$ и $Th(\mathbb{N}):=\set*{\varphi\:\middle\vert\:\mathbb{N}\models\varphi}$. Нестандартной моделью арифметики называется модель теории $Th(\mathbb{N})$, не изоморфная $\mathbb{N}$. Докажите, что существует счетная нестан-\linebreak дартная модель арифметики и опишите структуру порядка в такой модели.

	\item Будут ли (конечно) аксиоматизируемыми следющие классы структур (подходящей сигнатуры):

\begin{enumerate}
\item[(а)] всех групп; всех конечных групп; всех бесконечных групп; всех абелевых групп; всех циклических групп; всех групп без кручения;

\item[(б)] всех полей; всех конечных полей; всех полей фиксированной характеристики; всех бесконечных полей; всех алгебраически замкнутых полей; всех алгебраически замкнутых полей фиксированной характеристики;

\item[(в)] всех упорядоченных полей; всех конечных упорядоченных  полей; всех упорядоченных  полей фиксированной характеристики; всех вещественно замкнутых упорядоченных  полей.
\end{enumerate}

	\item Докажите, что любой аксиоматизируемый класс структур замкнут относительно элементарной эквивалентности и ультрапроизведений; верно ли обратное? 
Докажите, что если предложение логически следует из данного множества предложений, то оно логически следует из некоторого конечного подмножества этого множества.
Докажите,  что класс структур данной сигнатуры конечно аксиоматизируем в точности тогда, когда он сам и его дополнение (в классе всех структур этой сигнатуры) аксиоматизируемы.

	\item Докажите, что
\begin{enumerate}
\item[(а)] класс фундированных частичных порядков не аксиоматизируем ни в каком константном обогащении сигнатуры $\{\leq\}$, а класс нефундированных частичных порядков не аксиоматизируем в сигнатуре $\{\leq\}$, но аксиматизируем в ее константном обогащении.

\item[(б)] класс архимедовых упорядоченных полей не аксиоматизируем ни в каком константном обогащении его сигнатуры, а класс неархимедовых упорядоченных полей не аксиоматизируем в своей сигнатуре, но аксиматизируем в ее константном обогащении. 
\end{enumerate}

	\item Докажите, что любое элементарное расширение поля вещественных чисел, не изоморфное этому полю, содержит бесконечно большие и бесконечно малые элементы.
\end{enumerate}



\end{document}


