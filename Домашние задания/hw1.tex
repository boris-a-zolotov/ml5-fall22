\documentclass[a4paper,11pt]{article}
\usepackage{../ml5}

\begin{document}


\begin{center}
	\Large Домашнее задание 1. Логика предикатов
\end{center}

\begin{enumerate}
	\item Запишите аксиомы теории групп предложениями следующих сигнатур:
   \[\{\cdot\};\ \ \{\cdot,e\};\ \ \{\cdot,e,^{-1}\};\ \ \{f\}\]
	(последняя~— трехместный функциональный символ, обозначающий график ум-\linebreak ножения). Какая из этих сигнатур лучше соответствует изложению терии групп?

	\item Докажите, что если формула сигнатуры, состоящей из $n$ одноместных предикатных символов и не содержащей равенства, имеет модель, то она имеет модель мощности не более $2^n$. 

	\item Существует ли алгоритм, вычисляющей значение $\varphi^\mathbb{A}$ по любой конечной сигнатуре $\sigma$, $\sigma$-предложению $\varphi$, и конечной $\sigma$-структуре $\mathbb{A}$. Если да, оцените время работы вашего алгоритма.

	\item Докажите, что отношение $\leq$ не определимо в структуре $(\mathbb{Z};+)$, а операция $+$ не определима в структуре $(\mathbb{Z};\leq)$.

	\item Напишите предложения сигнатуры $\{\leq\}$ (с равенством) такие, что любая модель получившегося множества предложений является плотным линейным порядком с наименьшим, но без наибольшего элемента. Докажите, что любые две счетные модели этого множества  изоморфны. Верно ли это для моделей мощности континуум?
\end{enumerate}

\end{document}
