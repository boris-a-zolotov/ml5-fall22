\documentclass[a4paper,11pt]{article}
\usepackage{../ml5}

\begin{document}

   \newcommand{\enumsep}{\vspace{-2.8mm}
   		\begin{enumerate}[itemsep=0.4mm,leftmargin=2.5mm]}

\begin{center}
	{\Large Домашнее задание 12. Кодирование \(\text{ИП}_{\sigma}\), представимость функций.}

	{\it 23 ноября\ \(\to\)\ 30 ноября} \\
	{\it Последнее в разделе “Выводимость и (не)разрешимость”}
\end{center}

	Будем говорить, что функция \(f \lr*{x_1, \ldots, x_n}\) {\it представима} в минимальной арифметике, если существует формула \(\psi\lr*{\overline{x}, y}\) такая, что для всех \(x_1,\ldots,x_n \in \bn\)
	\[ \text{МА}\ \vdash\ \forall\,y\ \,\psi\lr*{\widehat{x_1}, \ldots, \widehat{x_n}, y}\ \ 
		\longleftrightarrow\ \ y = \overset{\hspace{-0.05cm}\tikz{\draw (0,0) -- (1.1,0.14) -- (2.2,0)}}{f\lr*{x_1,\ldots,x_n}}. \]

\begin{enumerate}
   \item Докажите, что следующие функции рекурсивны: \enumsep
	\item[(а)] $y!$~(факториал); $x^y$~(возведение в степень);
	\item[(б)] функция, перечисляющая без повторения простые числа в порядке возрастания;
	\item[(в)] последовательность Фибоначчи;
	\item[(г)] количество простых чисел, не превосходящих $y$. \end{enumerate}

   \item Докажите, что следующие множества рекурсивны: \enumsep
	\item[(а)] множество всех кодов термов сигнатуры $\sigma=\{<,+,\cdot,0,1\}$; 
	\item[(б)] множество кодов имен $\widehat{n}$ натуральных чисел; 
	\item[(в)] множество всех кодов формул сигнатуры $\sigma$.
	   \end{enumerate}

   \item Докажите рекурсивность функции $f$: \enumsep
	\item[(а)] \[ f(a,b) = \begin{cases} \text{код терма }t+s,& a,b\text{ — коды термов }t,s, \\ 0,& \text{иначе.} \end{cases} \]
	\item[(б)] \[ f(a) = \begin{cases} \text{код формулы }t+0=t,& a\text{ — код терма }t, \\ 0,& \text{иначе.} \end{cases} \]
	\item[(в)] \[ f(a) = \begin{cases} \text{код формулы }\neg\varphi,& a\text{ — код формулы }\varphi, \\ 0,& \text{иначе.} \end{cases} \]
	\item[(г)] \[ f(a,b,c) = \begin{cases} \text{код формулы }(\varphi\land\psi)\to\psi,& a,b,c\text{ — коды формул }\varphi, \psi, \theta, \\ 0,& \text{иначе.} \end{cases} \] \end{enumerate}

   \item Докажите, что функции $+$, $\cdot$, $I^n_k$~(проекция), $\ell$~(характеристическая функция предиката~\(<\)) представимы в минимальной арифметике.

   \item Докажите, что суперпозиция представимых функций представима, и что минимизация представимой функции представима.

\end{enumerate} \end{document}
