
\begin{center}
Задание 12. Рекурсивные определения, кодирование ИП$_\sigma$,  представимость РФ. 
\end{center}

1. Докажите, что следующие функции рекурсивны: $y!$; $x^y$; функция, перечисляющая без повторения простые числа в порядке возрастания; последовательность Фибоначчи; количество простых чисел, не превосходящих $y$. 
\medskip

2. Докажите, что следующие множества рекурсивны: 

множество всех кодов термов сигнатуры $\sigma=\{<,+,\cdot,0,1\}$; 

множество кодов имен $\hat{n}$ натуральных чисел; 

множество всех кодов формул сигнатуры $\sigma$;

множество всех кодов аксиом минимальной арифметики;

множество всех кодов логических следствий минмальной арифметики. 
\medskip

3. Докажите рекурсивность функции $f$:

 если $a$ и $b$ — коды термов $t$ и $s$ соответственно, то $f (a, b)$ --- код терма $t + s$, иначе $f (a, b) = 0$;

если $a$ --— код терма $t$, то $f (a)$ --- код формулы $t + 0 = t$, иначе $f (a) = 0$;

если $a$ --— код формулы $\varphi$, то $f (a)$ --- код формулы $\neg\varphi$, иначе $f (a) = 0$;

если $a, b, с$ --— коды формул $\varphi, \psi, \theta$ соответственно, то $f (a, b, с)$ --- код формулы $(\varphi\wedge\psi)\to\psi$, иначе $f (a, b, c) = 0$.
\medskip

4. Докажите, что функции $+,\cdot,l,I^n_k$ представимы в минимальной арифметике.
\medskip

5. Докажите, что суперпозиция представимых функций представима, и что минимизация представимой функции представима.






\end{document}


