\documentclass[a4paper,11pt]{article}
\usepackage{../ml5}

	\DeclareMathOperator{\rOp}{\text{\tt Оп}}
	\DeclareMathOperator{\rPr}{\text{\tt Прог}}
	\DeclareMathOperator{\rRe}{\text{\tt Рег}}
	\DeclareMathOperator{\rPam}{\text{\tt пам}}
	\DeclareMathOperator{\rDl}{\text{\tt дл}}
	\DeclareMathOperator{\rSos}{\text{\tt сос}}

\begin{document}

   \newcommand{\enumsep}{\vspace{-2.8mm}
   		\begin{enumerate}[itemsep=0.4mm,leftmargin=2.5mm]}

\begin{center}
	{\Large Домашнее задание 14. R-Вычислимость и рекурсивность.}

	{\it (30 ноября\ \(\to\)\ 7 декабря)}
\end{center}

\begin{enumerate}

	\item $\rOp(x)$ Пусть Оп($a$) означает, что $a$ является кодом некоторого оператора,  Прог($a$) означает, что $a$ является кодом некоторой программы, и Рег($i,a$) означает, что $a$ является кодом некоторой программы, содержащей переменную $r_i$.
\medskip

Докажите, что предикаты Оп, Прог, и Рег  рекурсивны.
\medskip

	\item Пусть дл$(a)=l+1$, если $a$ является кодом некоторой программы длины $l+1$, и дл$(a)=0$, если $a$ не является кодом программы. 

Пусть пам$(a)$ равно памяти программы с кодом $a$, если $a$ является кодом некоторой программы, и пам$(a)=0$ в противном случае. 

Пусть сос$(a,x_0,\ldots,x_n,t)=\langle s\rangle$, если $a$ является кодом программы $P$ и $s$ --- состояние $P$ в момент $t$ при вычислении $\varphi_P(\bar{x})$, и сос$(a,x_0,\ldots,x_n,t)=0$, если $a$ не является кодом программы. 
\medskip

Докажите, что функции дл, пам, и сос  рекурсивны.  
\medskip

	\item Докажите, что функция на множестве $\mathbb{N}$ R-вычислима тогда и только тогда, когда она рекурсивна.

Докажите, что частичная функция на множестве  $\mathbb{N}$ R-вычислима тогда и только тогда, когда она рекурсивна.
\medskip

	\item Пусть $g$ --- произвольня функция на $\mathbb{N}$. Функция $f$ {\em рекусивна относительно} $g$ (символически, $f\leq_Tg$), если $f$ получается из $g,+,\cdot,\chi_<,I^n_k$ последовательными применениями суперпозиции и минимизации.

Функция $f$ {\em R-вычислима относительно} $g$, если существует вычисляющая ее программа с оракулом $g$.
\medskip

Докажите аналог предыдущего упражнения для относительных (оракульных) вычислений.
\medskip

	\item Пусть $\varphi_n=\varphi^{(1)}_P$, если $n$ является кодом программы $P$ и $\varphi_n=\emptyset$, если $n$ не является кодом программы. Тогда $\varphi$ --- нумерация всех рекурсивных частичных функций, а $W_n=dom(\varphi_n)$ --- нумерация всех рекурсивно перечислимо множеств.

Докажите, что частичная функция $(n,x)\mapsto\varphi_n(x)$ рекурсивна, а множество $\{n\mid n\in W_n\}$ рекурсивно перечислимо, но не рекурсивно.




\end{enumerate}

\end{document}
