	% По-хорошему, все упоминаемые множества переделать бы в
	% \set*{ … \:\middle\vert\: … }

	% ЛевЕнгейма

\subsection*{Лекция 3}
% Закомментировал, потому что формулировка в конце предыдущей лекции
%\begin{stat}
%    $$\varphi([a_1], \ldots, [a_k]) \Longleftrightarrow \{i\mid \mathbb{A} \models \varphi(a_1(i), \ldots, a_k(i))\} \in F.$$ 
%:\end{stat} 

\begin{stat}[Следствие]
    Для ультрафильтра $F$ и предложения $\vfi$ выполенно
    $$\mathbb{A}_F \models \varphi \Longleftrightarrow \{i \mid \mathbb{A}_i \models \varphi\} \in F.$$
\end{stat} 

% TODO: надо переписать. Вероятно полностью.
\begin{proof} (теоремы Лося)
    Доказательство приведём индукцией по построению формулы. Простейшие формулы в виде предиката и равенства двух термов рассматриваются очевидно, это - база. Обратим внимание на функциональный символ $f \in \sigma$. Как он интерпретируется? $$f^{\mathbb{A}_F}([a_1], \ldots, [a_k]):= [\lambda_i f^{\mathbb{A}_i}(a_1(i), \ldots, a_k(i))]$$ 

    Из определения декартового у нас было $$f^{\mathbb{A}}([a_1], \ldots, [a_k]):= \lambda_i f^{\mathbb{A}_i}(a_1(i), \ldots, a_k(i)),$$ 

    где $i \mathbb{A}\mapsto f^{\mathbb{A}_i}(a_1(i), \ldots, a_k(i))$, и $\lambda x f(x) = f$. Причём согласно фильтру 

    \begin{equation*}
        \begin{aligned}
            a_1 &\equiv_F a_1' \\ 
            &\vdots \\ 
            a_k &\equiv_F a_k' \\ 
            f^{\mathbb{A}}(a_1, \ldots, a_k) &\equiv_F f^{\mathbb{A}}(a_1', \ldots, a_k').
        \end{aligned}
    \end{equation*} 

    $J_i \{i\mid a_1(i) = a_1'(i)\} \in F$, $f^{\mathbb{A}_i}(a_1(i), \ldots, a_k(i)) = J_1 \cap \ldots, \cap J_k \in F = f^{\mathbb{A}}(a_1', \ldots, a_k')$. Константы $c^{\mathbb{A}}$ интерпретируются как $\lambda_i c^{\mathbb{A}_i}$, переменные означиваются каким-то образом $x_j \mathbb{A}psto a_j(i)$, $t^{\mathbb{A}_i} = f^{\mathbb{A}_i}(t_1^{\mathbb{A}_i}, \ldots, t_k^{\mathbb{A}_i})$, значит, $t^{\mathbb{A}}(a_1, \ldots, a_k) = f^{\mathbb{A}}(t_1^{\mathbb{A}}(\overline{a}), \ldots, t_k^{\mathbb{A}}(\overline{a}))$. Соответственно, из определения это верно для простейших формул. Перейдём теперь к сложным формулам. \ 
    
    Более сложные формулы строятся из простых при помощи логических связок и кванторов. Достаточно рассматривать только конъюнкцию, отрицанию и существование (остальные выражаются через них). Пусть мы хотим проверить $$\mathbb{A}_F \models (\varphi \wedge \psi)(a_1, \ldots, a_k).$$ Это означает, что $\mathbb{A}_F \models \varphi([\overline{a}])$ и $\mathbb{A} \models \psi ([\overline{a}])$. $J = \{i \mid \mathbb{A}_i \models \varphi(\overline{a(i)})\} \in F$. Проверяется $i \in J \cap K$, $$\{\mathbb{A}_i \models (\varphi \wedge \psi)(a_1(i), \ldots, a_k(i))\} \in F.$$ Отрицание также легко проверяется для ультрафильтров, так как есть свойство дополнения. 

    \begin{equation*}
        \begin{aligned}
            \mathbb{A}_F &\models \neg \varphi([\overline{a}]) \\ 
            \neg (\mathbb{A}_F &\models \varphi([\overline{a}])) \\ 
            & \ldots
        \end{aligned}
    \end{equation*} 

    Существование проверяется следующим образом: 

    \begin{equation*}
        \begin{aligned}
            \varphi &= \varphi(x_1, \ldots, x_k), \\ 
            \varphi &= \exists x \theta (x, x_1, \ldots, x_k). \\ 
            \mathbb{A}_F &\models \varphi([a_1], \ldots, [a_k]), \\ 
            \mathbb{A}_F &\models \theta([b], [a_1], \ldots, [a_k]) \text{ для некоторого } b \in \mathbb{A}.  
        \end{aligned}
    \end{equation*} 

    И нам нужно доказать в две стороны. Для этого рассматриваем  

    \begin{equation*}
        \begin{aligned}
            J &= \set*{i\:\middle\vert\: \mathbb{A}_i \models \theta(b(i), a_1(i), \ldots, a_k(i))}, \\ 
            K &= \set*{i\:\middle\vert\: \mathbb{A}_i \models \varphi(a_1(i), \ldots, a_k(i))}.
        \end{aligned}
    \end{equation*} 

    Это~— элементы $F$, которые в разных случаях лежат друг в друге. Не уловил суть, надо будет дописать и переписать.
\end{proof} 

\begin{theorem}[Гёделя-Мальцева о компактности]
    Бесконечное множество предложений $\Gamma$ имеет модель тогда и только тогда, когда каждое его конечное подмножество $\Gamma$ имеет модель.
\end{theorem}  

\begin{proof}
    В одну сторону очевидно. Доказываем в обратную сторону.

    Пусть $I = \set*{i\:\middle\vert\: i \text{~— конечное подмножество } \Gamma}$.
    Из существования модели для каждого $i\in I$, по аксиоме выбора, существует семейство структур $\{\ba_i\}_{i\in I}$ такое, что $\forall i~\ba_i\models i$.

    % Рассмтрим декартово произведение $\mathbb{A} = \prod_i \mathbb{A}_i$ и $G_i = \set*{j \in I\:\middle\vert\: i \subseteq j}$. Если $k \in I$, то $G_i \cap G_k = G_{i \cup k}$ ($I$ - бесконечно). Утверждается, что $F = \set*{A \subseteq I\:\middle\vert\: \exists i (G_i \subseteq A)}$~— ультрафильтр. Свойства проверяются очевидно. 
   
    Будем строить ультрапроизведение, соответствующее декартовому произведению $\prod_{i\in I} \ba_i$, подходящее под требования теоремы. Начнём с построения фильтра. Для каждого $i\in I$ пусть $G_i = \set*{j \in I\:\middle\vert\: i \subseteq j}$. Для каждой пары $i, j\in I$ выполнено $G_i\cap G_k = G_{i\cup k}$. Пусть $F = \set*{A \subseteq I\:\middle\vert\: \exists i~G_i \subseteq A}$. Можно проверить, что это фильтр ($\emptyset$ отсутствует, так как все $G_i$ непусты; надмножество $A\supseteq G_i$ тоже содержит $G_i$; пересечение $A\supseteq G_i$ и $B\supseteq G_j$ содержит $G_{i\cup j}$). По доказанному ранее, существует ультрафильтр $H\supseteq F$.

    Наконец, рассмотрим ультрапроизведение $\ba_H$. Для любой формулы $\vfi\in\Gamma$, имеем $\{\vfi\}\in I$, поэтому $G_{\{\vfi\}}\in F\subseteq H$. По теореме Лося, так как $\forall i\in G_{\{\vfi\}}$ выполнено $\ba_i\models \vfi$, то и $\ba_H\models \vfi$.
\end{proof}

% Раздел 5
\subsubsection{Понижение и повышение мощности}

\begin{definition} \ 
    \begin{itemize}
        \item (Уже определялось выше) \emph{$\ba$ — подструктура $\bb$} (обозначается $\mathbb{A} \subseteq \mathbb{B}$), если $A\subseteq B$ и значения простых формул на элементах $\ba$ в $\mathbb{A}$ и $\mathbb{B}$ совпадают; 
        \item \emph{$\ba$ — элементарная подструктура $\bb$} (обозначается $\mathbb{A} \preceq \mathbb{B}$), если $A \subseteq B$ и значения любых формул на элементах $\mathbb{A}$ в $\ba$ и $\bb$ совпадают (то есть $\forall \overline{a}\in \ba$ выполнено $\vfi^\ba(\overline{a}) = \vfi^\bb(\overline{a})$); 
        \item \emph{$\ba$ элементарно эквивалентно $\bb$} (обозначается $\mathbb{A} \equiv \mathbb{B}$), если они удовлетворяют одни и те же предложения.
    \end{itemize}
\end{definition}

\begin{stat}
    $\mathbb{A} \preceq \mathbb{B}$, тогда $\mathbb{A} \subseteq \mathbb{B}$ и $\mathbb{A} \equiv \mathbb{B}$. 
\end{stat}

\begin{theorem}[Лёвенгейма-Сколема, понижение]
    Пусть есть $\mathbb{A}$, $X \subseteq A$, $|X| \leq |\text{For}_\sigma|$. Тогда существует $\mathbb{B} \preceq \mathbb{A}$: $X \subseteq B$ и $|\mathbb{B}| \leq |\text{For}_\sigma|$. 
\end{theorem}
