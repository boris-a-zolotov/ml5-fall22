\subsection{Лекция 7}

\begin{theorem} \  

    \begin{enumerate}
        \item $T$ -- модельно полна; 
        \item Для любой $\mathbb{A} \models T$, теория $T \cup D(\mathbb{A})$ полна; 
        \item (Тест Робинсона) Для люлых $\mathbb{A}, \mathbb{B} \models T$ ($\mathbb{A} \subseteq \mathbb{B}$, тогда любое $\Sigma_1$-предложение $\sigma_A$, которое истино в $\mathbb{B}$, будет истино и в $\mathbb{A}$); 
        \item $\Sigma_1 = \Pi_1$ по модулю $T$, то есть любая $\Sigma_1$-формула $\varphi(\overline{x})$ равносильно подходит $\Pi_1$-формуле $\psi(\overline{x})$ в $T$ ($T \models \forall \overline{x} (\varphi(\overline{x}) \leftrightarrow \psi (\overline{y}))$); 
        \item Любая формула $\varphi(\overline{x})$ равносильна подходящей $\Pi_1$-формуле в $T$.
    \end{enumerate}
\end{theorem}

\begin{proof}
    На практиках мы уже доказали $1 \Rightarrow 2$, $2 \Rightarrow 3$. Нетривиальным является следствие $3 \Rightarrow 4$. Пусть $\varphi(\overline{y})$ -- $\Sigma_1$-формула. Нам нужно найти $\Pi_1$-формулу $\psi(\overline{y})$ $T \models \forall \overline{y} (\varphi{\overline{y}} \leftrightarrow \psi(\overline{y}))$. Скажем, $\overline{y} = (y_1, \ldots, y_k)$, а $\overline{c} = (c_1, \ldots, c_k)$ -- новые константы в обогащённое сигнатуре. $T \models \varphi(\overline{c}) \leftrightarrow \psi(\overline{c})$. 

    \[ 
        \Gamma = \{\gamma \in \Pi_1 \vert T \models \varphi(\overline{c}) \rightarrow \gamma\}.
    \]

    Достаточно доказать, что $T \cup \Gamma \models \varphi(\overline{c})$ ($T \cup \{\gamma_1, \ldots, \gamma_m\} \models \varphi$, $\psi = \gamma_1 \wedge \ldots \wedge \gamma_k \in \Pi_1$). Для любой $\mathbb{A} \models T \cup \Gamma$, $\mathbb{A} \models \varphi$, имеет ли $T \cup \{\varphi\} \cup D(\mathbb{A})$ модель? Предоложим противное, так конечное $T \cup \{\varphi\} \cup D(\mathbb{A})$ не имеет модели. $\mathbb{A} \models \exists \overline{x} \delta(\overline{x})$ ($\delta = \delta_1 \wedge \ldots wedge \delta_m$, $\delta_i = \delta_i(d_1, \ldots, d_m)$, $\delta_i \in D(\mathbb{A})$). $T \cup \{\varphi\} \models \forall \overline{x} \neg \delta(\overline{x})$, а значит, 

    \[ 
        T \models (\varphi \rightarrow \forall \overline{x} \neg \delta(\overline{x})) \text{\rotatebox[origin=c]{180}{$\models$}} \mathbb{A},
    \]

    противоречие. Тут стоит дописать $4 \Rightarrow 5$. $5 \Rightarrow 1$ Нам нужно, чтобы $\mathbb{A} \subseteq \mathbb{B} \Rightarrow \mathbb{A} \preceq \mathbb{B}$, где $\mathbb{A}$ и $\mathbb{A}$ -- модели $T$. $\varphi(\overline{x}) \equiv_T \psi(\overline{x})$, $\overline{a} \in \mathbb{A}$, $\varphi^{\mathbb{A}}(\overline{a}) = \varphi(\overline{a})$. Значит, 

    \[ 
        \psi(\overline{x}) = \forall \overline{y} \theta(\overline{x}, \overline{y})
    \]

    из утверждения в этом пункте. Тогда $\mathbb{B} \models \varphi(\overline{a}), \psi(\overline{a})$, $\mathbb{A} \models \psi(\overline{a}), \varphi(\overline{a})$. 
\end{proof} 

\begin{prop}[Модельно полных теорий] \ 

    \begin{enumerate}
        \item Любая модельно полная теория $\Pi_2$-аксиоматизируемая;
        \item (Тест Линдстрёма) Если теория $\Pi_2$-аксиоматизируема, не имеет конечныз моделей и категорична в некоторой мощности $\lambda \geq |\text{For}_\sigma|$, то она модельно полна; 
        \item Если модельго полная теория $T$ имеет модель, котоаря вкладывается в любую модель $T$, то $T$ -- полная; 
        \item Если любые две модели модельгно полной $T$ можно вложить в 
    \end{enumerate}
\end{prop}

\begin{proof} \

    \begin{enumerate}
        \item $T$ -- модель полная. Достаточно доказать, что $\text{Mod}(T)$ замкнут относительно объединения цепей (теорема Чэна-Лося-Сушко) 
         
        \[ 
            \mathbb{A}_0 \subseteq \mathbb{A}_1 \subseteq \ldots \mathbb{A} = \bigcup_n \mathbb{A}_n,
        \]

        где $\mathbb{A}_i \models T_i$. Пракда ли, что $\mathbb{A} \models T$? Из модельной полноты $\mathbb{A}_0 \preceq \mathbb{A}_1 \preceq \ldots$, $\mathbb{A}_n \preceq \mathbb{A}_i$, тогда $T \text{\rotatebox[origin=c]{180}{$\models$}} \mathbb{A}_n \equiv \mathbb{A}$. 

        \item Остаётся на совесть юных читателей! (доказательство не было закончено)
        \item $\mathbb{A}$ -- структура, изоморфная подструктуре любой модели $\mathbb{B} \models T$, $\forall \mathbb{B} (\mathbb{A} \preceq \mathbb{B})$. Для любой $\varphi$ $T \models \varphi \vee T \models \neg \varphi$, откуда и получим, что из $\mathbb{A}$ следует либо эта формула, либо её отрицание. 
        \item Совсем не уловил\ldots 
    \end{enumerate}
\end{proof}