\subsection{Лекция 8}

\begin{stat}[Тест Линдстрёма]
    Если некоторая теория $T$ $\Pi_2$-аксиоматизируема, не имеет конечных моделей и категорична в некоторой мощности $\lambda\ge|\text{For}_\sigma|$, то она модельно полна.
\end{stat}

\begin{proof}
    Достаточно проверить выполнение теста Робинсона. Предположим противное. Тогда существуют структуры $\ba\subseteq\bb$, являющиеся моделями $T$, и $\Sigma_1$-формула $\vfi$ сигнатуры $\sigma$ такая, что $\bb\models\vfi(\overline{a})$, но $\ba\models\neg\vfi(\overline{a})$ для некоторого набора $\overline{a}\in A$.

    Рассмотрим обогащение исходной сигнатуры $\sigma^+ = \sigma\cup\set*{P}$, где $P$ — новый одноместный предикатный символ. Структура $\bb^+$ — обогащение $\bb$ до $\sigma^+$-структуры, в которой $P$ интерпретируется как множество $\ba$. Пусть $T^+ = \text{Th}(\bb^+) \supseteq T$.

    Также заметим, что $\theta\in T^+$, где
    $$
    \theta = \exists x_1\, \dots \exists x_k\, (P(x_1)\wedge\dots\wedge P(x_k)\wedge\vfi(\overline{x})\wedge\neg\vfi^P(\overline{x})).
    $$
    Здесь $\vfi^P$ — \emph{релятивизация} формулы $\vfi$ относительно предиката $P$, то есть все кванторы формулы ограничены $P$ (более формально определяется по индукции: $(\forall y\,\psi)^P\defeq \forall y\, (P(y)\to\psi^P)$ и $(\exists y\,\psi)^P\defeq \exists y\,(P(y)\wedge \psi^P)$). Формула $\theta$ как раз утверждает, что для некоторого $\overline{a}\in A$ $\bb\models\vfi(\overline{a})$, но $\ba\models\neg\vfi(\overline{a})$.

    (Примечание записывающего: дальше доказательство даётся в форме наброска, как на лекции. Формальное доказательство длиннее)

    По теоремам Лёвенгейма-Сколема (надо использовать обе), теория $T^+$ имеет модель $\mathbb{D}$ мощности $\lambda$ такую, что мощность множества $P^{\mathbb{D}^+}$. Тогда $\mathbb{D} = \mathbb{D}^+|_\sigma$ — модель теориия $T$. Пусть $\bc$ — подструктура $\mathbb{D}$ на множестве $P^{D^+}$. Тогда $\bc$ — модель теории $T$ мощности $\lambda$.

    \begin{stat}
        Для любой $\bc\models T$ мощности $\lambda$ существует экзистенциально замкнутое расширение $\bc^+\supseteq\bc$ ($\bc^+\models T$) мощности $\lambda$.
    \end{stat}
    \begin{definition}
        Структура $\bc^+$ называется \emph{экзистенциально замкнутой}, если для любого расширения $\mathbb{E}\supseteq\mathbb{C}^+$ для любой $\Sigma_1$-формулы $\vfi$ и набора $\overline{c}\in \bc^+$ из истинности $\vfi(\overline{c})$ в $\mathbb{E}$ следует истинность в $\bc^+$.
    \end{definition}

    Идея доказательства утверждения — рассмотреть цепочку $\bc\subseteq\bc_0\subseteq \bc_1\subseteq \dots$ моделей $T$, где на каждом шаге к $\bc_n$ добавляются все решения экзистенциальных формул с коэффициентами в $\bc_n$. Тогда каждое $\bc_i$ имеет мощность $\lambda$, тогда $\bigcup \bc_i$ — экзистенциально замкнуто, модель $T$ и имеет мощность $\lambda$.

    Тогда получаем противоречие, так как из $\bc\subseteq \mathbb{D}\models\vfi(\overline{a})$ следует $\bc\models \vfi(\overline{a})$ из категоричности.
\end{proof}




\subsubsection{Элиминация кванторов}
\begin{definition}
    Теория $T$ допускает элиминацию кванторов, если любая формула равносильна бескванторной.
\end{definition}

\begin{prop}\
    \begin{enumerate}
        \item Если $T$ допускает элиминацию кванторов, то она модельно полна;

        \item Если для любой бескванторной формулы $\theta(\overline{x}, y)$ формула $\exists y\, \theta(\overline{x}, y)$ равносильна некоторой бескванторной, то $T$ допускает элиминацию кванторов;

        \item Если для данной формулы $\varphi$ выполнено условие $\circledast$, то $\varphi$ равносильна бескванторной $\psi(\overline{x})$ в $T$.

            Условие $\circledast$: для любых вложений (то есть изоморфизмов на подструктуры) $f: \mathbb{C} \rightarrow \mathbb{A}$, $g: \mathbb{C} \rightarrow \mathbb{B}$ $\sigma$-структуры $\mathbb{C}$ в модели $\mathbb{A}, \mathbb{B}$ теории $T$ и для любых значений $\overline{c}\in\mathbb{C}$  выполнено $\varphi(f(\overline{c}))=\varphi(g(\overline{c}))$  

        \item Пусть для любой бескванторной $\theta(\overline{x}, y)$ формула $\phi(\overline{x}) = \exists y~\theta(\overline{x}, y)$ удовлетворяет $\circledast$. Тогда $T$ допускает элиминацию кванторов.
    \end{enumerate}
\end{prop}


\begin{proof}\
    \begin{enumerate}
    % Пункт 1
        \item Одно из эквивалентных условие модельно полной теории~— равносильность любой формулы универсальной, а любая бескванторная формула лежит в $\Pi_1$.

    % Пункт 2
        \item % По утверждению лектора, доказательство было на практике
        Доказывается индукцией по сложности формулы, кванторы всеобщности можно представить через кванторы существования.

    % Пункт 3
   \item
       Схема доказательства уже неоднократно нами применялась.

       Зафиксируем $\varphi(x_1, \ldots, x_k)$, удовлетворяющую $\circledast$. Обогатим сигнатуру новыми константами $\overline{d} = (d_1, \ldots, d_k)$. Для доказательства нам надо придумать бескванторную $\psi$, такую что в исходной сигнатуре $T\models \forall \overline{x}~(\varphi(\overline{x})\leftrightarrow\psi(\overline{x}))$, что эквивалентно $T\models \varphi(\overline{d}) \leftrightarrow \psi(\overline{d})$ в обогащённой структуре.

       Пусть $\Gamma = \{\gamma \text{ - бескванторное }\sigma_{\overline{d}} \text{-предложение}  \mid T\models \varphi(\overline{d})\rightarrow \gamma\}$. Достаточно доказать, что $T\cup\Gamma\models\varphi(\overline{d})$. Действительно, если это так, то по теореме о компакности для конечного $\{\gamma_i\}_i\subset\Gamma$ верно $T\cup\{\gamma_i\}_i\models \varphi(\overline{d})$, откуда для $\gamma=\wedge_i\gamma_i$ верно $T\models \gamma\rightarrow \varphi(\overline{d})$.

       Будем доказывать от противного. Тогда $T\cup\Gamma\cup\{\neg\varphi(\overline{d})\}$ имеет модель $\mathbb{A}$. Обозначим через $d_i'$ интерпретацию $d_i$ в структуре $\mathbb{A}$.

       Пусть $\mathbb{C}$~— подструктура $\mathbb{A}$, порождённая элементами $d_1',\ldots, d_k'$ (то есть кроме этих элементов есть ещё все применения функций к этим переменным, а предикаты как в исходной структуре). Конечно, $\mathbb{C}$ не обязана быть моделью $T$. Пусть $f:\mathbb{C}\rightarrow\mathbb{A}$~— тождественное вложение.
        
       $\text{Diag}(\mathbb{C})$~— вариант $D(\mathbb{C})$, но без использования новых констант — в нашем случае уже есть имена для всех элементов (термы от $d_i'$), поэтому новых символов добавлять не нужно. 

       \begin{stat} 
        $T\cup\text{Diag}(\mathbb{C})\cup\{\varphi(\overline{d})\}$ имеет модель.
       \end{stat}
       \begin{proof}
           Доказываем от противного. Тогда для некоторого конечного подмножества $\Gamma$ $T\cup\{\delta_1(\overline{d}),\ldots,\delta_n(\overline{d}))\}\cup\{\varphi(\overline{d})\}$ не имеет модели. Значит $T\models \bigwedge\delta_i(\overline{d})\rightarrow\neg\varphi(\overline{d})$, откуда по контрпозиции $T\models \varphi(\overline{d})\rightarrow\bigvee\neg\delta_i(\overline{d})$.
           
           Тогда $\gamma=\bigvee\neg\delta_i(\overline{d})$ лежит в $\Gamma$. Значит $\mathbb{A}\models \gamma$, но по определению $\text{Diag}$ выполнено $\mathbb{A}\models \neg\gamma$. Противоречие
       \end{proof}
        
       Пусть $\mathbb{B}'$~— модель, удовлетворяющая утверждению. Рассмотрим её обеднение $\mathbb{B}$ до структуры $\sigma$. Существует единственное вложение $g: \mathbb{C}\rightarrow \mathbb{B}$, переводящее $d_i$ в $d_i'$.

       Наконец, воспользуемся условием $\circledast$. По определению, в $\mathbb{A}$ $\varphi(f(\overline{d}))$ ложно, а в $\mathbb{B}$ $\varphi(g(\overline{d}))$ истинно. Противоречие.

    % Пункт 4
    \item Очевидно следует из пунктов 2 и 3.
    \end{enumerate}
\end{proof}


Тарский доказал, что структуры $(\mathbb{R}, <, +, \cdot, 0, 1)$ и $(\mathbb{C}, +, \cdot, 0, 1)$ допускают элиминацию кванторов. Доказательство конструктивное, но длинное. Мы докажем то же самое утверждение для всех алгебраически замкнутых полей, но не конструктивно.
% А можно доказывать и конструктивно, но доказательство сложнее.

\begin{exmpl}
    Теория ACF (теория алгебраически замкнутых полей) допускает элиминацию кванторов
\end{exmpl}

% TODO: расписать доказательство
\begin{proof}(плохо записано)\

    Сначала докажем, что ACF модельно полна, используя тест Линдстрёма: $\Pi_2$ аксиоматизируемость по определению, ACF не имеет конечных моделей. 

    (TODO) С доказательством категоричности возникли проблемы, из-за существования полей разных характеристик. Но, насколько я понял, для полей фиксированной характеристики всё нормально, поэтому мы доказали модельную полноту $ACF_p$ ($p$ — характеристика). Вроде бы и сама ACF модельно полна, но это надо доказывать не тестом Линдстрёма.

    Для доказательства основного факта воспользуемся пунктом 4 предыдущего утверждения, обозначения те же. Для $\bc$ с вложениями $f, g$  в $\ba, \bb$ можно рассмотреть поле частных $\bc^*$ и его алгебраическое замыкание $\tilde{\bc}$, для которого можно придумать вложения $\tilde{f}, \tilde{g}$ в $\ba, \bb$, чтобы в диаграмме со стрелочками $f, g, \tilde{f}, \tilde{g}$ и вложением $\bc$ в $\tilde{\bc}$ они согласовывались. Тогда из модельной полноты, если $\vfi$ от образа верно в $\ba$, то и в $\bc$ и в $\bb$ тоже верно.
\end{proof}
