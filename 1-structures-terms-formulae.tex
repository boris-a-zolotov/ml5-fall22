\subsubsection{Структура}

Бурбаки классифицировал структуры как:
\begin{enumerate}[itemsep=-1.4mm]
	\item операции,
	\item частичные порядки,
	\item топологические структуры.
\end{enumerate}

Последние не имеют приложения в логике~— их мы рассматривать не будем. “Операции”~— это структуры алгебраические, “частичные порядки”~— это структуры, снабжённые каким-либо отношением.

\begin{definition}
	\emph{Сигнатура}~— набор функциональных, предикатных и константных символов вместе с функцией, задающей арность этих символов.

	Функциональные символы интерпретируются как функции $A^n \to A$, предикатные символы~— как функции $A^m \to \{\text{и}; \text{л}\}$, а константы~— как элементы $A$ (или, что равносильно, функции $\{\varnothing\} \to A$).

	Будем называть \emph{$\sigma$-структурой} (\emph{структурой сигнатуры $\sigma$}) пару $(A, I)$, где $A$~— непустое множество, а $I$~— интерпретация сигнатурных символов $\sigma$ в $A$.
\end{definition}

\begin{exmpl}
	Сигнатура упорядоченного кольца~— $\langle {+},\ {\cdot}\,;\ {<};\ 0,\ 1 \rangle$. Можно добавить вычитание и взятие противоположного, но они выражаются в имеющейся сигнатуре.
\end{exmpl}

\begin{definition}
	$\mathbb{A}$, $\mathbb{B}$~— $\sigma$-структуры. Тогда отображение $\varphi: \mathbb{A} \to \mathbb{B}$ называется гомоморфизмом, если оно задаёт $\varphi: A \to B$, что для всякой функции $f^n$ из сигнатуры $\sigma$ и для всяких $a_1, \dots, a_n \in A$
	\[\varphi(f_A(a_1, \dots, a_n)) = f_B(\varphi(a_1), \dots, \varphi(a_n)),\]
	для всякого предиката $P^m$ в сигнатуре $\sigma$ и всяких $a_1, \dots, a_m \in A$
	\[P_A(a_1, \dots, a_m) \quad \Longrightarrow \quad P_B(\varphi(a_1), \dots, \varphi(a_m))\]
	и для всякой константы $c$ сигнатуры $\sigma$
	\[\varphi(c_A) = c_B.\]

	$\varphi$~— изоморфизм, если $\varphi$~— гомоморфизм, биективен, и $\varphi^{-1}$~— гомоморфизм.

	$\mathbb{A}$ называется \emph{подструктурой} $\mathbb{B}$ ($\mathbb{A} \subseteq \mathbb{B}$), если $A \subseteq B$ и $\varphi: A \to B, a \mapsto a$~— гомоморфизм.
\end{definition}

\subsubsection{Термы и формулы}

\begin{definition}
	Фиксируем некоторое множество $V$~— \emph{“множество переменных”}~— символы $\wedge$, $\vee$, $\to$, $\neq$ и символы $\forall x$ и $\exists x$ для всякого $x \in V$.
	
	\emph{Терм}~— это понятие, рекурсивно определяемое следующими соотношениями:
	\begin{itemize}
		\item переменная~— терм,
		\item константа~— терм,
		\item для всяких термов $t_1, \dots, t_n$ и функции $f^n$ выражение $f(t_1, \dots, t_n)$~— терм.
	\end{itemize}

	\emph{Формула}~— это понятие, рекурсивно определяемое следующими соотношениями:
	\begin{itemize}
		\item для всяких термов $t_1$, $t_2$ выражение $t_1 = t_2$~— формула,
		\item для всяких предиката $P^n$ из $\sigma$ и термов $t_1, \dots, t_n$ выражение $P(t_1, \dots, t_n)$~— формула,
		\item для всяких формул $\varphi$ и $\psi$ выражения $\varphi \wedge \psi$, $\varphi \vee \psi$, $\varphi \to \psi$, $\neg \varphi$~— формулы,
		\item для всяких формулы $\varphi$ и переменной $x$ выражения $\forall x \varphi$ и $\exists x \varphi$~— формулы.
	\end{itemize}
	$\Formu_\sigma$~— множество всех формул с сигнатурой $\sigma$.
\end{definition}

\begin{exmpl}
	В кольцах всякий терм можно свести к полиному с целыми коэффициентами. В мультипликативных группа~— моному с целым коэффициентов.
\end{exmpl}

\begin{task}
	Семейства термов и формул задаются контекстно свободными грамматиками.
\end{task}

\begin{definition}
	Переменная $x$ называется \emph{свободной} в формуле $\varphi$, если есть вхождение $x$ не покрывается никаким квантором $\forall x$ и никаким квантором $\exists x$. $\FV(\varphi)$~— множество всех свободных переменных формулы $\varphi$.
\end{definition}

\subsubsection{Значение термов и формул}

\begin{definition}
	Пусть $t$~— терм в сигнатуре $\sigma$, а $\mathbb{A}$~— $\sigma$-структура. Тогда $t^{\mathbb{A}}: A^n \to A$~— \emph{означивание $t$}, некоторая функция, полученная подставлением вместо констант их значений в $\mathbb{A}$ и последующим рекурсивным означиванием по синтаксическому дереву $t$. Аналогично получается означивание формулы $f^{\mathbb{A}}: A^n \to \{\text{и}; \text{л}\}$.
\end{definition}
