\subsection*{Лекция 13}

\begin{prop}[свойства кодирования]\
    \begin{enumerate}
    \item Разным термам и формулам соответствуют разные коды
    \item Существует алгоритм, вычисляющий по данному логическому объекту (терму, формуле) его код
    \item Наоборот: существует алгоритм, вычисляющий по коду этот объект.
    \item Предикаты $\text{Т}, \text{Ф}, \text{Ф}_0, \text{Пр}, \ldots$ и функции $\text{отр}, \text{подс}$ рекурсивны.
    \item Если множество формул $T$ рекурсивно (то есть рекурсивен предикат $P_T(x)$), то $\text{Выв}_T(a, b)$ рекурсивен.
    \end{enumerate}
\end{prop}

\begin{proof}
    Второй и третий пункт не являются строгими математическими утверждениями, потому то понятие алгоритма не математично. Но нестрого утверждение понятно. % классное доказательство
    Доказательство пятого пункта есть в книжке\footnote{В.Л.Селиванов -- <<Краткий курс математической логики>> 1992 года издания}. Остальные пункты доказываются несложно % классное доказательство №2
\end{proof}


%%%%%%%%%%%%%%%
% раздел 14
\subsubsection{Представимость $\text{ИП}_\sigma$ в минимальной арифметике}
Напомним, что MA (минимальная арифметика) состоит из 10 аксиом в сигнатуре $\sigma = \{<, +, \cdot, 0, 1\}$. Каждому натуральному числу $n$ можно сопоставить терм $\hat{n}$ по следующим правилам: $\hat{0} = 0$, $\hat{1}=1$, $\widehat{n+1} = (\hat{n})+1$

\begin{definition}
    Предикат $P(x_1,\ldots,x_n)$ на $\bn$ называется представимым в MA, если сущесвует формула $\vfi(x_1,\ldots,x_n)$ такая, что для любых значений $\overline{x}\in\bn$ выполнено $P(\overline{x})=\text{И} \Rightarrow \text{MA} \vdash \vfi(\widehat{x_1},\ldots,\widehat{x_n})$ и $P(\overline{x})=\text{Л}\Rightarrow \text{MA}\vdash \neg\vfi(\widehat{x_1},\ldots,\widehat{x_n})$

    Функция $f(x_1,\ldots,x_n)$ называется представимой в MA, если сущесвует формула $\vfi(\overline{x}, y)$ такая, что для любых $\overline{x}\in\bn$ $\text{MA}\vdash \forall y~(\psi(\widehat{x_1},\ldots,\widehat{x_n})\leftrightarrow y=\widehat{f(\overline{x} )})$
\end{definition}

\begin{theorem}
    Любой рекурсивный предикат представим в MA.
\end{theorem}
\begin{proof}
    TODO %TODO
\end{proof}


%%%%%%%%%%%%%%%%%
% раздел 15
\subsubsection{Неразрешимость и неполнота арифметики. Проблемы разрешимости}

\begin{theorem}[Чёрча о неразрешимости арифметики]
    Для любой непротиворечивой теории $T\supseteq MA$ множество $[T]=\{\vfi\mid T\vdash \vfi\}$ не рекурсивно. 
\end{theorem}
\begin{proof}
    TODO %TODO
\end{proof}

\begin{theorem}[Гёделя о неполноте арифметики]
    Любая непротиворечивая рекурсивная теория $T\supseteq MA$ неполна.
\end{theorem}
\begin{proof}
    TODO %TODO
\end{proof}

\begin{theorem}
    Множество тождественно истинных формул неразрешимо (нерекурсивно)
\end{theorem}
