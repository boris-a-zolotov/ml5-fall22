\subsection{Лекция 5} 

\begin{stat}[следствие теоремы Лёвенгейма—Сколема] \ 
     
    \begin{enumerate}
        \item\label{skolem_sled1} Если $\sigma$-теория имеет бесконечную модель, то она имеет модель любой мощности хотя бы $|\text{For}_\sigma|$; 
        \item\label{skolem_sled2} Если $\sigma$-теория имеет конечные модели сколь угодно большой мощности, то она имеет модель любой мощности хотя бы $|\text{For}_\sigma|$. 
    \end{enumerate}
\end{stat}

\begin{proof}
    В пункте \ref{skolem_sled1} сначала берём модель $\bb$ очень большой мощности (как в доказательстве теоремы о повышении с использованием теоремы о компактности). Потом, по теореме о понижении мощности находим элементарную подструктуру $\bc\preceq\bb$ мощности $|\text{For}_\sigma|$. TODO: Лектор не закончил доказательство, отвлёкшись на следующую теорему.
\end{proof} 

\begin{theorem}[без доказательства]
    Логика предикатов~— единственная логика, для которой верны и теорема о компактности и теорема о понижении мощности.
\end{theorem}

%\subsection{Аксиоматизированные классы}
% Это раздел 6 в нумерации
\subsection{Аксиоматизируемые классы структур}

\begin{definition} \  
$\text{Sent}_\sigma \supseteq T$, $\text{Str}_\sigma \supseteq K$. Сопоставим $T \mapsto \text{Mod}(T)$, $K \mapsto \text{Th}(K) = \{\varphi \mid \forall \mathbb{A}\in K (\mathbb{A}\models \varphi)\}$.
    \begin{enumerate}
        \item  Класс $K$ называется \textit{аксиоматизируемым}, если $K = \text{Mod}(T)$ для всех $T$; 
        \item $K$~— \textit{конечно аксиоматизируемый}, если $K = \text{Mod}(T)$ для некоторого конечного $T = \{\varphi_1, \ldots, \varphi_n\}$. Это равносильно аксиоматизируемости одной формулой ($\varphi_1 \wedge \ldots \wedge \varphi_n$)
    \end{enumerate}
\end{definition}

\begin{prop} \ 

    \begin{enumerate}
        \item \label{aksiom1} $T \subseteq T'$, тогда $\text{Mod}(T) \supseteq \text{Mod}(T')$; 
        \item \label{aksiom2} $K \subseteq K'$, тогда $\text{Th}(K) \supseteq Th(K')$; 
        \item \label{aksiom3} $K \subseteq \text{Mod}(\text{Th}(K))$ и  $T \subseteq \text{Th}{\text{Mod(T)}}$; 
        \item \label{aksiom4} Любое пересечение аксиоматизируемых классов является аксиоматизируемым классом. Объединение двух аксиоматизируемых классов является аксиоматизируемым классом; 
        \item \label{aksiom5} Класс $K$ является аксиоматизируемым тогда и только тогда, когда $K = \text{Mod}(\text{Th}(K))$; 
        \item \label{aksiom6} $K$ конечно аксиоматизируемый тогда и только тогда, когда $K$ и $\text{Str}_\sigma \backslash K$ аксиоматизируемы; 
        \item \label{aksiom7} $K$~— аксиоматизируемый тогда и только тогда, когда $K$ замкнут относительно $\equiv$ и ультрапроизведений.
    \end{enumerate}
\end{prop}

\begin{proof}\
    Свойства \ref{aksiom1}, \ref{aksiom2}, \ref{aksiom3} и \ref{aksiom5} очевидны. Свойство \ref{aksiom6} давалось на практику.

    В пункте \ref{aksiom4} для $\{K_i\}_{i\in I}$ с $K_i=\text{Mod}(T_i)$ выполнено $\bigcap K_i = \text{Mod}(\bigcup T_i)$. Для $K=\text{Mod}(T)$, $K'=\text{Mod}(T')$ выполнено $K\bigcup K' = \text{Mod}(\{\vfi\vee\psi \mid \vfi\in T, \psi\in T'\})$

    Докажем \ref{aksiom7} в левую сторону (в правую~— д/з). Пусть $\{\mathbb{A}\}_{i \in I} \in K$, тогда $\MA_F \in K$. Проверим, что $K$ совпалает с множеством $\text{Mod}(\text{Th}(K))$ (этого достаточно по пункту \ref{aksiom5}), причём из свойства \ref{aksiom3} включение $K$ в множество моделей очевидно, а с другим придётся повозиться.
    
    Пусть $\mathbb{A}\models \text{Th}(K)$, и нам нужно показать, что $\mathbb{A}\in K$. $\mathbb{A}\equiv \mathbb{B}_{F^*}$, где $F^*$~— некий ультрафильтр на подходящем множестве $I$; $B_i \in K$. Возьмём $I\defeq \text{Th}(A)$. Утверждается, что для любого $\varphi \in \text{Th}(\mathbb{A})$ существует $\mathbb{B} \in K$ такой, что $\mathbb{B} \models \varphi$.  

    Возьмём любое $\varphi$ и предположим, что это не верно. То есть, для любой структуры $\mathbb{B} \in K$, $\mathbb{B} \models \neg \varphi$, тогда $\neg \varphi \in \text{Th}(K)$ и в $\mathbb{A}$ истино $\neg \varphi$~— противоречие. Таким образом $\varphi \mapsto \mathbb{B}_\varphi \models \varphi$, и мы получили семейство структур. Надо построить ультрафильтр.  
    
    Для каждого $\varphi \in I$ рассмотрим $U_{\varphi} := \{\psi \in \text{Th}(\mathbb{A}) | \models \psi \rightarrow \varphi\}$ (то есть импликации $\psi\to\vfi$ тождественно истинны) $\varphi \in U_\varphi$, поэтому непусто. $U_{\varphi} \cap U_{\varphi'} = U_{\varphi \wedge \varphi'} \neq \emptyset$ и принадлежит $\text{Th}(\mathbb{A})$. Пусть $F = \{J \subseteq \text{Th}(\mathbb{A}) \mid \exists \varphi\, (J \supseteq U_\varphi)\}$. Это фильтр. Пусть $F^*$~— любой ультрафильтр, расширяющий $F$. Проверим, что $\ba\equiv B_{F^*}$.

    Пускай для некоторого предложения $\vfi$ выполнено $\mathbb{A}\models \varphi$, по определению $\varphi \in I = \text{Th}(\mathbb{A})$, $\mathbb{B}_\varphi \models \varphi$. Хотим доказать
    \[ 
        U_{\varphi} \subseteq \{\psi \in \text{Th}(\mathbb{A}) \mid  \mathbb{B}_{\psi} \models \varphi\} \in F \subseteq F^*.
    \]
    Действительно, для всех $\psi\in U_\phi$ выполнено $\models \psi\to\vfi$, в частности $\bb_\psi\models \psi\to\vfi$. Также $\bb_\psi\models\psi$, поэтому $\bb_\psi\models\vfi$. Значит, по теореме Лося, $B_{F^*}\models \vfi$.
\end{proof}

%%%%%%

\begin{definition}[Иерархия по числу перемен кванторов]\
    % $\Sigma_n$ ($n \in \mathbb{N}$)~— множество $\sigma$-формул (равносильных): 
    \begin{itemize}
        \item $\Sigma_0$~— все формулы, равносильные бескванторным формулам; 
        \item $\Sigma_1$~— формулы, равносильные формулам вида $\exists \overline{x}\, \psi(\overline{x}, \overline{y})$, где $\psi$~— бескванторная; 
        \item $\Sigma_2$~— формулы, равносильные формулам вида $\exists \overline{x_1}\, \forall \overline{x_2}\, \psi(\overline{x_1}, \overline{x_2}, \overline{y})$, где $\psi$~— бескванторная;
        \item и так далее по \textit{иерархии $\sigma$-формул по числу перемен кванторов в предварённой нормальной форме} получаем $\{\Sigma_n\}_{n \in \mathbb{N}}$.
    \end{itemize}

    $\Pi_n$~—  определяется аналогично с заменой $\exists$ на $\forall$ и наоборот.
\end{definition}

% Определения предела, непрерывности лежат в \Pi_3, его понимают все математики. Класс \Pi_4 не понимает почти никто!

\begin{prop} \ 

    \begin{itemize}
        \item $\Sigma_n \cup \Pi_n \subseteq \Sigma_{n+1} \cap \Pi_{n+1}$; 
        \item $\varphi \in \Pi_n$ тогда и только тогда, когда $\neg \varphi \in \Sigma_n$; 
        \item $\bigcup \Sigma_n = \bigcup \Pi_n = \text{For}_\sigma$.
    \end{itemize}
\end{prop}

\begin{theorem}
    Аксиоматизируемый класс является $\Pi_1$-аксиоматизируемым (или \emph{универсально аксиоматизируемым}) тогда и только тогда, когда он замкнут относительно подструктур (то есть если какая-то структура лежит в классе, то и любая её подструктура тоже лежит в нём).
\end{theorem}
\begin{remark}
    Этот вопрос решён для любого класса иерархии. Но общий вид этой теоремы весьма труден.
\end{remark}
\begin{proof}
    Докажем слева направо. Пусть у нас есть класс $K = \text{Mod}(T)$, где $T$~— множество $\Pi_1$ предложений. Нам нужно доказать, что он замкнут относительно подструктур.
    %Если $\mathbb{A}\subseteq \mathbb{B} \models T$, то утверждается, что $\mathbb{A}\models T$. По-другому это можно расписать как $A \subseteq \mathbb{B} \models \varphi = \forall \overline{x} \psi(\overline{x})$. И это верно, потому что очевидно. Начнём с конца: $\mathbb{A}\models \psi(\overline{a})$ при $\overline{a} \in A$, тогда $\mathbb{B} \models \psi(\overline{a})$. 
    Пусть $\bb\models T$, а $\ba\subseteq\bb$. Зафиксируем $\Pi_1$-предложение $\vfi=\forall\overline{x}\,\psi(\overline{a})$ из $T$. $\ba\models\vfi$ означает, что для любого набора $\overline{b}\in\bb$ выполнено $\psi(\overline{b})$, откуда в частности для любого набора $\overline{b}\in\ba$ выполнено $\psi(\overline{b})$. Значит $\ba\models\vfi$, откуда $\ba\models T$.


    Справа налево. $K = \text{Mod}(T)$ для некоторой теории $T$, введём класс аксиом $\Gamma = \{\varphi \in \Pi_1\text{-предложения} \mid T \models \varphi\}$. Оказывается, что $K = \text{Mod}(\Gamma)$, докажем это. Включение $K$ в $\text{Mod}(\Gamma)$ очевидно. В другую~— возьмём некоторую модель множества $\Gamma$ ($\mathbb{B} \models \Gamma$), тогда нужно проверить, что $\mathbb{B} \in K$, Конечно, нужно воспользоваться замкнутостью. Достаточно найти $\mathbb{C} \in K$, что $\mathbb{B} \subseteq \mathbb{C}$. 

    \begin{definition}
        Если что, $\text{Th}(\mathbb{A}) = \{\varphi \mid \mathbb{A} \models \varphi\}$, $\Phi \subseteq \text{Sent}_\sigma$, $\text{Th}_{\Phi}(\mathbb{A}) = \{\varphi \in \Phi | \mathbb{A} \models \varphi\}$.
    \end{definition}
    Утверждается, что существует $\mathbb{A} \models T$ такая, что $\text{Th}_{\Sigma_1}(\mathbb{A}) \supseteq \text{Th}_{\Sigma_1}(\mathbb{B})$.
    % В точности нам нужно доказать, что для $T \cup \text{Th}_{\Sigma_1}(\mathbb{B})$ имеется модель.
    В качестве такого $\ba$ возьмём модель $T \cup \text{Th}_{\Sigma_1}(\mathbb{B})$, существование которой мы докажем по теореме о компактности.
    Предположим, что $T \cup \{\psi_1, \ldots, \psi_n\}$ не имеет модели. $\psi = \psi_1 \wedge \ldots \wedge \psi_n \in \Sigma_1$, $T \cup \{\psi\}$ не имеет модели, значит $T \models \neg \psi \in \Pi_1$, а значит, $\mathbb{B} \models \neg \psi$. По определению $\neg\psi\in\Gamma$, $\bb\models\Gamma$, поэтому $\bb\models\neg\psi$. Но с другой стороны $\mathbb{B} \models \psi$, противоречие. \\ 

    Нам нужно вложить $\mathbb{B} \subseteq \mathbb{C} \models T$. Это всё равно, что найти модель для  $T \cup D(\mathbb{B})$ . Применим в очередной раз теорему о компактности. То есть хочется, чтобы $T \cup \{\delta_1, \ldots, \delta_m\}$ имело модель, где $\delta_i = \delta_i(\overline{c})$ ($c \in \sigma_B$). Возьмём какие-то новые переменные и подставим: $\mathbb{B} \models \exists \overline{x}\, (\delta_1(\overline{x}) \wedge \ldots \wedge \delta_m(\overline{x}))$. Это предложение истинно в $\mathbb{B}$, лежит в $\Sigma_1$, а значит, истинно и в $\mathbb{A}$. Тогда при подходящей интерпретации $\mathbb{A}$~— искомая модель.
\end{proof}
